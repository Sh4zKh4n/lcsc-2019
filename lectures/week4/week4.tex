% Created 2019-02-21 Thu 14:16
% Intended LaTeX compiler: pdflatex
\documentclass[10pt]{beamer}
\usepackage[utf8]{inputenc}
\usepackage[T1]{fontenc}
\usepackage{graphicx}
\usepackage{grffile}
\usepackage{longtable}
\usepackage{wrapfig}
\usepackage{rotating}
\usepackage[normalem]{ulem}
\usepackage{amsmath}
\usepackage{textcomp}
\usepackage{amssymb}
\usepackage{capt-of}
\usepackage{hyperref}
\usepackage{minted}
\usepackage{tikz}
\definecolor{cwiRed}{HTML}{BF1238}
\definecolor{cwiBlue}{HTML}{0B5D7D}
\setbeamertemplate{footline}[text line]{%
\parbox{\linewidth}{\noindent\vspace*{2pt}\noindent\rule{\linewidth}{0.4pt}\\{\scriptsize\noindent\vspace*{7pt}\insertshortauthor\hfill\insertshorttitle\hfill\insertdate}}
}
\renewcommand*\footnoterule{}
\renewcommand{\vec}[1]{\mathbf{#1}}
\usepackage{lmodern}
\usetheme[progressbar=head]{metropolis}
\author{WISM454 Laboratory Class Scientific Computing, Jan-Willem Buurlage}
\date{\today}
\title{The C++ STL}
\hypersetup{
 pdfauthor={WISM454 Laboratory Class Scientific Computing, Jan-Willem Buurlage},
 pdftitle={The C++ STL},
 pdfkeywords={},
 pdfsubject={},
 pdfcreator={Emacs 26.1 (Org mode 9.1.14)}, 
 pdflang={English}}
\begin{document}

\maketitle

\section{C++}
\label{sec:org2aad055}
\begin{frame}[fragile,label={sec:orgb7de52c}]{Standard libraries}
 C++ library

\begin{itemize}
\item \texttt{<iostream>}
\item \texttt{<array>}
\item \texttt{<vector>}
\item \texttt{<string>}
\item \texttt{<algorithm>}
\item \ldots{}
\end{itemize}

C library (ports)

\begin{itemize}
\item \texttt{<cstdint>}
\item \texttt{<cmath>}
\item \ldots{}
\end{itemize}
\end{frame}

\begin{frame}[fragile,label={sec:org57cc487}]{\texttt{<iostream>}}
 \begin{minted}[frame=none,xleftmargin=\parindent]{cpp}
#include <iostream>

std::cout << "Hello, world!\n";

int a;
std::cin >> a;
\end{minted}
\end{frame}

\begin{frame}[fragile,label={sec:org28f256a}]{Arrays in C++}
 \begin{minted}[frame=none,xleftmargin=\parindent]{cpp}
int[5] xs = {1, 2, 3, 4, 5};
\end{minted}
\begin{itemize}
\item Size has to be a compile time constant
\item C-style dynamic (i.e. at runtime) arrays by heap allocation
\end{itemize}

\begin{minted}[frame=none,xleftmargin=\parindent]{cpp}
int size;
std::cin >> size;
int* xs = new int[size];
// ... use xs
delete[] xs;
\end{minted}
\end{frame}

\begin{frame}[fragile,label={sec:org48994e4}]{\texttt{std::array}}
 \begin{minted}[frame=none,xleftmargin=\parindent]{cpp}
#include <array>

auto xs = std::array<int, 5>();

for (auto& x : xs) {
    x = 3;
}
\end{minted}
\begin{itemize}
\item Standard statically (compile-time and constant) sized container
\end{itemize}
\end{frame}

\begin{frame}[fragile,label={sec:org3dbcc02}]{\texttt{std::vector}}
 \begin{minted}[frame=none,xleftmargin=\parindent]{cpp}
#include <vector>

auto ys = std::vector<int>(5);

for (auto& y : xs) {
    y = 3;
}
\end{minted}

\begin{itemize}
\item Dynamically (runtime and non-constant) sized container
\end{itemize}
\end{frame}

\begin{frame}[fragile,label={sec:org6337afd}]{Usage}
 \begin{minted}[frame=none,xleftmargin=\parindent]{cpp}
void histogram(const std::vector<uint32_t>& xs,
    int bin_count = 10) {
    auto bins = std::vector<uint32_t>(bin_count);
    // ... fill bins
    for (auto& bin : bins) {
        // ... output bins
    }
}

struct lcrng_with_state {
    std::array<uint32_t, 10> state;
};
\end{minted}
\end{frame}

\begin{frame}[fragile,label={sec:org8b99dcb}]{\texttt{std::vector}}
 \begin{minted}[frame=none,xleftmargin=\parindent]{cpp}
auto xs = std::vector<int>(5);
auto y = xs[3]; // ~> 0
xs.empty(); // ~> false
xs.size(); // ~> 5
xs.capacity(); // ~> ?

xs.push_back(5);
xs.size(); // ~> 6

xs.clear();
xs.size(); // ~> 0
\end{minted}
\end{frame}


\begin{frame}[fragile,label={sec:org334fcc6}]{\texttt{std::pair}, \texttt{std::tuple}}
 \begin{minted}[frame=none,xleftmargin=\parindent]{cpp}
#include <utility>

std::pair<int, int> xs = std::make_pair(3, 3);
std::tuple<int, int, int> ys = std::make_tuple(3, 3, 5);
// xs.first, xs.second
// std::get<0>(ys), std::get<1>(ys)
\end{minted}
\end{frame}

\begin{frame}[fragile,label={sec:orgbf3f088}]{\texttt{<numeric>}: `iota, accumulate`}
 \begin{itemize}
\item Many algorithms in \texttt{<numeric>} operate on containers at once
\item Does not have to be a vector, can also be a stack, queue or C-style array
\end{itemize}

\begin{minted}[frame=none,xleftmargin=\parindent]{cpp}
// make a vector with 100 elements
auto xs = std::vector<int>(100);

// fill with 0 .. 99
std::iota(xs.begin(), xs.end(), 0);

// sum all the elements up
auto sum = std::accumulate(xs.begin(), xs.end(),
                           0, std::plus<int>());
// -> sum = 4950
\end{minted}
\end{frame}

\begin{frame}[fragile,label={sec:orgeaa9034}]{\texttt{<numeric>}: \texttt{partial\_sum, adjacent\_difference, inner\_product}}
 \begin{minted}[frame=none,xleftmargin=\parindent]{cpp}
// partial sum (in-place)
std::partial_sum(xs.begin(), xs.end(),
                 xs.begin());

// -> xs = [0, 1, 3, 6, ...]
std::adjacent_difference(xs.begin(), xs.end(),
                         xs.begin());
// -> xs = [0, 1, 2, 3, ...]

auto alpha = std::inner_product(xs.begin(), xs.end(),
                   xs.begin(), 0);
// -> alpha = 0 * 0 + 1 * 1 + 2 * 2 + 3 * 3 + ...
// -> alpha = 328350
\end{minted}
\end{frame}

\begin{frame}[fragile,label={sec:orga496a6f}]{\texttt{<algorithm>}: \texttt{transform}}
 \begin{itemize}
\item In \texttt{<algorithm>}, many more operations on container are defined, e.g.
\end{itemize}

\begin{minted}[frame=none,xleftmargin=\parindent]{cpp}
std::transform(xs.begin(), xs.end(), xs.begin(),
    [](auto i) { return i * i; });
// -> xs = [0, 1, 4, 9, ...]

auto alpha = std::accumulate(xs.begin(), xs.end(), 0);
// -> alpha = 328350
\end{minted}
\end{frame}

\begin{frame}[fragile,label={sec:org13f3d10}]{\texttt{<algorithm>}: \texttt{fill}, \texttt{generate}}
 \begin{minted}[frame=none,xleftmargin=\parindent]{cpp}
std::fill(xs.begin(), xs.end(), -1);
// -> xs = [-1, -1, -1, ..., -1]

int n = 1;
std::generate(xs.begin(), xs.end(),
              [&n](){ return n++; });
// -> xs = [1, 2, 3, ..., 100]
\end{minted}
\end{frame}

\begin{frame}[fragile,label={sec:orge5658ed}]{\texttt{<algorithm>}: \texttt{any\_of}}
 \begin{minted}[frame=none,xleftmargin=\parindent]{cpp}
// -> xs = [1, 2, 4, 6, ..., 100]
auto any = std::any_of(xs.begin(), xs.end(),
                       [](auto x) { x % 2 == 1 });

auto any_shift = std::any_of(xs.begin() + 1, xs.end(),
                             [](auto x) { x % 2 == 1 });
// -> any = true, any_shift = false
\end{minted}
\end{frame}


\begin{frame}[fragile,label={sec:org9209f80}]{\texttt{<algorithm>}: \texttt{all\_of}}
 \begin{minted}[frame=none,xleftmargin=\parindent]{cpp}
// -> xs = [1, 2, 4, 6, ..., 100]
auto all = std::all_of(xs.begin(), xs.end(),
                       [](auto x) { x % 2 == 0 });

auto all_shift = std::all_of(xs.begin() + 1, xs.end(),
                             [](auto x) { x % 2 == 0 });
// -> all = false, all_shift = true
\end{minted}
\end{frame}

\begin{frame}[fragile,label={sec:orgbcb9b90}]{\texttt{<algorithm>}: \texttt{none\_of}}
 \begin{minted}[frame=none,xleftmargin=\parindent]{cpp}
// -> xs = [1, 2, 4, 6, ..., 100]
auto none = std::none_of(xs.begin(), xs.end(),
                         [](auto x) { x % 2 == 1 });

auto none_shift = std::none_of(xs.begin() + 1, xs.end(),
                               [](auto x) { x % 2 == 1 });
// -> none = false, none_shift = true
\end{minted}
\end{frame}

\begin{frame}[fragile,label={sec:org2169c98}]{\texttt{<algorithm>}: `reverse, sort, partial\(_{\text{sort}}\)`}
 \begin{minted}[frame=none,xleftmargin=\parindent]{cpp}
// -> xs = [1, 2, 4, 6, ..., 100]
std::reverse(xs.begin(), xs.end());

// -> xs = [100, ..., 6, 4, 2, 1]
std::sort(xs.begin(), xs.end());

// -> xs = [1, 2, 4, 6, ..., 100]
std::iota(xs.begin(), xs.end(), xs.begin(), 0);
std::reverse(xs.begin(), xs.end());
// -> xs = [50, 49, 48, ... 0], 

std::partial_sort(xs.begin(), xs.begin() + 3, xs.end());
// -> xs = [0, 1, 2, ?, ..., ?]
\end{minted}
\end{frame}
\begin{frame}[fragile,label={sec:orgab69d25}]{Other containers}
 \begin{itemize}
\item \texttt{std::stack}: FIFO container
\item \texttt{std::queue}: LIFO container
\item \texttt{std::dequeue}: fast insertion at begin and end
\item \texttt{std::list}: linked list
\item \texttt{std::priority\_queue}: retrieve largest element in \(\mathcal{O}(1)\), insert in
\(\mathcal{O}(log(n))\)
\item \texttt{std::map}: key-value dictionary, usually red-black tree
\item \texttt{std::unordered\_map}: key-value dictionary using hash map. \(\mathcal{O}(1)\)
insert, delete, find -- but not sorted
\end{itemize}
\end{frame}
\end{document}